\documentclass[tikz,border=10pt]{standalone}

% Essential packages for TikZ
\usepackage{tikz}
\usepackage{graphicx}
\usepackage{amsmath}
\usepackage{calc}
\usepackage{makecell}

% Additional TikZ libraries that might be needed
\usetikzlibrary{positioning,calc,arrows.meta}

% Define the \scalebox command if not using the graphicx package already
\usepackage{graphics}

% Latin
\newcommand{\eg}{\textit{e.g.}}
\newcommand{\ie}{\textit{i.e.}}
\newcommand{\cf}{\textit{cf.}}
\newcommand{\etc}{\textit{etc.}}
\newcommand{\etal}{\textit{et~al.}}

% Math
\newcommand{\mathbold}[1]{\bm{#1}}
\newcommand{\mbf}[1]{\mathbf{#1}}
\newcommand{\vect}[1]{\mathbf{#1}}
\newcommand{\vectb}[1]{\bm{#1}}
\newcommand{\T}{^\mathsf{T}}
\newcommand{\mat}[1]{\mathbf{#1}}
\newcommand{\kron}{\raisebox{1pt}{\ensuremath{\:\otimes\:}}} % Kronceker product
\newcommand{\bigO}{\mathcal{O}}

% Custom macros
%\newcommand{\T}{\top}    % Transpose
\newcommand{\dd}{\,\mathrm{d}} % E.g. \int f(x) \dd x
\newcommand{\E}{\mathbb{E}}    % Expectation
\newcommand{\R}{\mathbb{R}}    % Real numbers
\newcommand{\N}{\mathrm{N}}   % Gaussian distribution
\DeclareMathOperator{\tr}{tr}
\DeclareMathOperator{\diag}{diag}
\DeclareMathOperator{\chol}{chol}
\DeclareMathOperator{\dchol}{dchol}
\DeclareMathOperator{\Cov}{Cov}
\DeclareMathOperator{\Var}{Var}
\DeclareMathOperator{\gammad}{Gamma}
\DeclareMathOperator{\expd}{Exp}
\DeclareMathOperator{\sech}{sech}
%\DeclareMathOperator{\U}{U}
\DeclareMathOperator{\argmin}{arg\,min}
\DeclareMathOperator{\argmax}{arg\,max}
\newcommand{\KL}[2]{\mathrm{D}_\mathrm{KL}\left[#1\|#2\right]}

% Bold Greek symbols
\newcommand{\valpha}[0]{\mathbold{\alpha}}
\newcommand{\vbeta}[0]{\mathbold{\beta}}
\newcommand{\vsigma}[0]{\mathbold{\sigma}}
\newcommand{\vchi}[0]{\mathbold{\chi}}
\newcommand{\vepsilon}[0]{\mathbold{\varepsilon}}
\newcommand{\veta}[0]{\mathbold{\eta}}
\newcommand{\vmu}[0]{\mathbold{\mu}}
\newcommand{\vomega}[0]{\mathbold{\omega}}
\newcommand{\vxi}[0]{\mathbold{\xi}}
\newcommand{\vphi}[0]{\mathbold{\phi}}
\newcommand{\vtheta}[0]{\mathbold{\theta}}
\newcommand{\vTheta}[0]{\mathbold{\Theta}}
\newcommand{\vzeta}[0]{\mathbold{\zeta}}
\newcommand{\MPsi}[0]{\mathbold{\Psi}}
\newcommand{\MPhi}[0]{\mathbold{\Phi}}
\newcommand{\MSigma}[0]{\mathbold{\Sigma}}
\newcommand{\MTheta}[0]{\mathbold{\Theta}}
\newcommand{\invchisq}[0]{\mathrm{Inv\text{-}}\chi^2}
\renewcommand{\mid}{\,|\,}
\newcommand{\imag}[0]{\mathrm{i}}
\def\x{\mathbf{x}}
\def\y{\mathbf{y}}
\def\G{\mathbf{G}}
\def\Q{\mathbf{Q}}
\def\R{\mathbf{R}}
\def\bR{\mathbb{R}}
\def\e{\mathbf{e}}
\def\X{\mathbf{X}}
\def\H{\mathbf{H}}
\def\h{\mathbf{h}}
\def\Y{\mathbf{Y}}
\def\E{\mathbf{E}}
\def\P{\mathbf{P}}
\def\p{\mathbf{p}}
\def\N{\mathcal{N}}
\def\U{\mathbf{U}}
\def\enc{\boldsymbol{\varphi}}

% Use these macros for big roman symbols
\newcommand{\RL}{\mathrm{L}}
\newcommand{\RM}{\mathrm{M}}
\newcommand{\RQ}{\mathrm{Q}}
\newcommand{\RS}{\mathrm{S}}
\newcommand{\RU}{\mathrm{U}}

% Use these macros for roman symbols
\newcommand{\rb}{\mathrm{b}}
\newcommand{\rc}{\mathrm{c}}
\newcommand{\rrm}{\mathrm{m}}
\newcommand{\ro}{\mathrm{o}}
\newcommand{\rs}{\mathrm{s}}
\newcommand{\rrq}{\mathrm{q}}

% Use these macros for vectors and matrices
\newcommand{\va}{\mbf{a}}
\newcommand{\vb}{\mbf{b}}
\newcommand{\vc}{\mbf{c}}
\newcommand{\vd}{\mbf{d}}
\newcommand{\ve}{\mbf{e}}
\newcommand{\vf}{\mbf{f}}
\newcommand{\vg}{\mbf{g}}
\newcommand{\vh}{\mbf{h}}
\newcommand{\vi}{\mbf{i}}
\newcommand{\vj}{\mbf{j}}
\newcommand{\vk}{\mbf{k}}
\newcommand{\vl}{\mbf{l}}
\newcommand{\vm}{\mbf{m}}
\newcommand{\vn}{\mbf{n}}
\newcommand{\vo}{\mbf{o}}
\newcommand{\vp}{\mbf{p}}
\newcommand{\vq}{\mbf{q}}
\newcommand{\vr}{\mbf{r}}
\newcommand{\vs}{\mbf{s}}
\newcommand{\vu}{\mbf{u}}
\newcommand{\vv}{\mbf{v}}
\newcommand{\vw}{\mbf{w}}
\newcommand{\vx}{\mbf{x}}
\newcommand{\vy}{\mbf{y}}
\newcommand{\vz}{\mbf{z}}
\newcommand{\MA}{\mbf{A}}
\newcommand{\MB}{\mbf{B}}
\newcommand{\MC}{\mbf{C}}
\newcommand{\MD}{\mbf{D}}
\newcommand{\MF}{\mbf{F}}
\newcommand{\MG}{\mbf{G}}
\newcommand{\MH}{\mbf{H}}
\newcommand{\MI}{\mbf{I}}
\newcommand{\MJ}{\mbf{J}}
\newcommand{\MK}{\mbf{K}}
\newcommand{\ML}{\mbf{L}}
\newcommand{\MM}{\mbf{M}}
\newcommand{\MP}{\mbf{P}}
\newcommand{\MQ}{\mbf{Q}}
\newcommand{\MR}{\mbf{R}}
\newcommand{\MS}{\mbf{S}}
\newcommand{\MT}{\mbf{T}}
\newcommand{\MV}{\mbf{V}}
\newcommand{\MW}{\mbf{W}}
\newcommand{\MX}{\mbf{X}}
\newcommand{\MY}{\mbf{Y}}


% Random matrices
\def\mA{{\bm{A}}}
\def\mB{{\bm{B}}}
\def\mC{{\bm{C}}}
\def\mD{{\bm{D}}}
\def\mE{{\bm{E}}}
\def\mF{{\bm{F}}}
\def\mG{{\bm{G}}}
\def\mH{{\bm{H}}}
\def\mI{{\bm{I}}}
\def\mJ{{\bm{J}}}
\def\mK{{\bm{K}}}
\def\mL{{\bm{L}}}
\def\mM{{\bm{M}}}
\def\mN{{\bm{N}}}
\def\mO{{\bm{O}}}
\def\mP{{\bm{P}}}
\def\mQ{{\bm{Q}}}
\def\mR{{\bm{R}}}
\def\mS{{\bm{S}}}
\def\mT{{\bm{T}}}
\def\mU{{\bm{U}}}
\def\mV{{\bm{V}}}
\def\mW{{\bm{W}}}
\def\mX{{\bm{X}}}
\def\mY{{\bm{Y}}}
\def\mZ{{\bm{Z}}}

% Document begins
\begin{document}
    \begin{tikzpicture}

    % Styles
    \tikzstyle{label}=[font=\tiny]  
    \tikzstyle{arr}=[->,-latex,line width=1pt]  
    
    % The background image    
    \node[anchor=south west,inner sep=0] (image) at (0,0)
      {\includegraphics{fig/comparing_alignment.pdf}};
    
    % Scope for drawing on image
    \begin{scope}[x={(image.south east)},y={(image.north west)}]

    Annotations at the top of the figure
    \node (reaction) at (0.5,1.04) {};
    % \node[label] at ($(reaction) + (0,0.05)$) {\scalebox{1.}{\footnotesize An example identity reaction}};
    % \node[label] at ($(reaction) + (0,0.00)$) {\scalebox{1.}{\footnotesize \& denoiser outputs}};
    \node[label] at ($(reaction) + (-0.3,-0.02)$) {\scalebox{1.}{\scriptsize Reactant}};
    \node[label] at ($(reaction) + (0.3,-0.02)$) {\scalebox{1.}{\scriptsize Product}};

    % Perm eq. GNN output
    \node (c0) at (0.155,.47) {};
      \node[label] at ($(c0) + (-0.02,0)$) {\scalebox{1.}{$D_\theta(\X_T,\!\Y)^\mathcal{N}$}};
      \node[label] at ($(c0) + (-0.09,0.28)$) {\scalebox{1.}{C}};
      \node[label] at ($(c0) + (-0.028,0.28)$) {\scalebox{1.}{O}};
      \node[label] at ($(c0) + (0.028,0.28)$) {\scalebox{1.}{N}};

    % Perm eq GNN text
    \node (permGNN) at (0.41,0.62) {\scalebox{0.8}{Perm.Eq.}};
    \node (permGNN) at ($(permGNN) + (0,-0.04)$) {\scalebox{0.8}{GNN}};

    % The permutation equivariant GNN input box
    \node (c0) at (0.705,.47) {};
      \node[label] at ($(c0) + (0,0)$) {\scalebox{1.}{$\X_T^\mathcal{N}$}};
      \node[label] at ($(c0) + (-0.08,0.29)$) {\scalebox{1.}{C}};
      \node[label] at ($(c0) + (-0.035,0.29)$) {\scalebox{1.}{O}};
      \node[label] at ($(c0) + (0.004,0.29)$) {\scalebox{1.}{N}};
      \node[label] at ($(c0) + (0.045,0.29)$) {\scalebox{1.}{$\perp$}};

    \node (c0) at (0.90,.47) {};
      \node[label] at ($(c0) + (0,0)$) {\scalebox{1.}{$\Y^\mathcal{N}$}};
      \node[label] at ($(c0) + (-0.08+0.04,0.29)$) {\scalebox{1.}{C}};
      \node[label] at ($(c0) + (-0.04+0.04,0.29)$) {\scalebox{1.}{O}};
      \node[label] at ($(c0) + (0.00+0.04,0.29)$) {\scalebox{1.}{N}};

    % Aligned GNN output
    \node (c0) at (0.155,.05) {};
      \node[label] at ($(c0) + (0.05,0)$) {\scalebox{1.}{$D_\theta(\X_T,\!\Y,\!\P^{\Y\to\X})^\mathcal{N}$}};
      \node[label] at ($(c0) + (-0.05,0.29)$) {\scalebox{1.}{C}};
      \node[label] at ($(c0) + (-0.01,0.29)$) {\scalebox{1.}{O}};
      \node[label] at ($(c0) + (0.028,0.29)$) {\scalebox{1.}{N}};

    % Aligned GNN text
    \node (permGNN) at (0.39,0.2) {\scalebox{0.8}{Aligned}};
    \node (permGNN) at ($(permGNN) + (0,-0.04)$) {\scalebox{0.8}{GNN}};

    % The aligned GNN input box
    \node (c0) at (0.67,.17) {};
      \node[label] at ($(c0) + (0,0)$) {\scalebox{0.8}{$\X_T^\mathcal{N}$}};
      \node[label] at ($(c0) + (-0.055,0.18)$) {\scalebox{0.8}{C}};
      \node[label] at ($(c0) + (-0.025,0.18)$) {\scalebox{0.8}{O}};
      \node[label] at ($(c0) + (0.003,0.18)$) {\scalebox{0.8}{N}};
      \node[label] at ($(c0) + (0.025,0.18)$) {\scalebox{0.8}{$\perp$}};

    \node (c0) at (0.8,.05) {};
      \node[label] at ($(c0) + (0,0)$) {\scalebox{0.8}{$\P^{\Y\to\X}$}};

    \node (c1) at (0.92,.18) {};
      \node[label] at ($(c1) + (0,0)$) {\scalebox{0.8}{$\Y^\mathcal{N}$}};
      \node[label] at ($(c1) + (-0.028,0.175)$) {\scalebox{0.8}{C}};
      \node[label] at ($(c1) + (-0.005,0.175)$) {\scalebox{0.8}{O}};
      \node[label] at ($(c1) + (0.02,0.172)$) {\scalebox{0.8}{N}};

    \end{scope}
    \end{tikzpicture}
\end{document}